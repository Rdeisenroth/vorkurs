\documentclass[accentcolor=tud3c,colorbacktitle,12pt]{tudexercise}
\usepackage[T1]{fontenc}
\usepackage[utf8]{inputenc}
\usepackage[ngerman]{babel}
\usepackage{hyperref}
\usepackage{ifthen}
\usepackage{listings}
\usepackage{graphicx}
\usepackage{multicol}
\usepackage{multirow}

\setlength{\parindent}{0pt}

\begin{document}
\title{Programmierchallenge Wintersemester 2019/20 \\ {\small der Fachschaft Informatik}}
\subtitle{Wintersemester 2019/20}
\maketitle 
	
	\section*{TicTacToe \footnote[1]{\url{https://de.wikipedia.org/wiki/Tic-Tac-Toe}}}
	\subsection*{Ablauf des Spiels}
	Es wird ein Spielfeld generiert, welches aus 9 Feldern besteht \\
	
	Die Spieler setzen immer abwechselnd ein Feld in dem sie es mit ihrem Zeichen markieren bis entweder keine Felder mehr frei sind oder ein Spieler eine Reihe aus drei Feldern hat. \\
	
	
	\section{Das Spiel}
	\subsection*{Die Aufgabe}
	Programmiert in Python eine Spieladaption des oben beschriebenen TicTacToe. Dieses soll auf und in der Konsole funktionieren. Hierbei soll dss Spielfeld nach jedem Zug neu angezeigt werden, sprich immer wenn der Spieler an der Reihe ist. Anschließend muss der*die Spieler*in mittels der Konsole ein Feld auswählen, welches dann mit seinem Zeichen markiert wird. \\
	Das Programm soll selbst erkennen, wann das Spiel für einen der Spieler gewonnen/verloren ist bzw. wann es ein Unentschieden gibt. \\
	Weiter soll das Programm selbst gegen den Menschen spielen. Dabei ist es egal ob er Zufallszüge macht oder intelligent erkennt welchen Zug er spielen muss um ein Unentschieden zu erreichen.  
	
	\subsection*{Rahmen}
	Es existiert kein Rahmen oder Framework. Das Projekt besitzt außer diesem Dokument keine weiteren Unterlagen. Bei Fragen könnt ihr euch am besten an die Tutoren oder an die Orga wenden. Bitte haltet euch an das KISS-Prinzip\footnote[2]{\url{https://de.wikipedia.org/wiki/KISS-Prinzip}} (Keep it simple, stupid), sprich versucht eine möglichst einfache Lösung zu erstellen. Es muss auch kein Wunderwerk der Technik sein. Dennoch sind kreative Ideen gerne gesehen.
	
	\subsection*{Die Abgabe} 
	Es gibt zwei Möglichkeiten der Abgabe: Bis spätestens 14:00 Uhr am Do (03.10.2017) könnt ihr eure .py Datei in Moodle hochladen oder ihr schickt uns eine Mail an vorkurs@d120.de. Dort hängt ihr die Datei bitte als Anhang an. Dabei gilt die Ankunftszeit bei uns. 
	
\end{document}
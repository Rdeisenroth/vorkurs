\documentclass[accentcolor=3c,colorbacktitle,12pt]{tudaexercise}
\usepackage[T1]{fontenc}
\usepackage[utf8]{inputenc}
\usepackage[ngerman]{babel}
\usepackage{hyperref}
\usepackage{ifthen}
\usepackage{listings}
\usepackage{graphicx}
\usepackage{multicol}
\usepackage{multirow}

\setlength{\parindent}{0pt}

\begin{document}
\title{Programmierchallenge Wintersemester 2019/20 \\ {\small der Fachschaft Informatik}}
\subtitle{Wintersemester 2019/20}
\maketitle 
	
	\section*{TicTacToe \footnote[1]{\url{https://de.wikipedia.org/wiki/Tic-Tac-Toe}}}
	\subsection*{Ablauf des Spiels}
	Es wird ein Spielfeld generiert, welches aus 9 Feldern besteht. \\
	
	Die Spieler*innen markieren immer abwechselnd ein leeres Feld mit individuellen Zeichen (z.B. \texttt{X}, \texttt{O}). Das Spiel ist beendet, sobald keine Felder mehr frei sind oder ein*e Spieler*in eine Reihe oder Diagonale aus drei Feldern markiert hat, was den Sieg bedeutet. \\
	
	
	\section*{Das Spiel}
	\subsection*{Die Aufgabe}
	Programmiert in Python eine Spieladaption des oben beschriebenen TicTacToe. Dieses soll auf und in der Konsole funktionieren. Hierbei soll das Spielfeld nach jedem Zug neu angezeigt werden. Anschließend muss der*die Spieler*in mittels der Konsole ein Feld auswählen, welches dann mit seinem Zeichen markiert wird. \\
	Das Programm soll selbst erkennen, wann das Spiel für eine*n der Spieler*innen gewonnen oder verloren ist bzw. wann es ein Unentschieden gibt. Im Anschluss zeigt das Programm den Ausgang des Spiels an und beendet sich. \\
	Weiter soll das Programm selbst gegen den Menschen spielen. Dabei ist es egal, ob das Programm Zufallszüge macht oder intelligent erkennt welchen Zug es spielen muss, um ein Unentschieden oder Sieg zu erreichen.  
	
	\subsection*{Rahmen}
	Es existiert kein Rahmen oder Framework. Das Projekt besitzt außer diesem Dokument keine weiteren Unterlagen. Bei Fragen könnt ihr euch am besten an die Tutor*innen oder an die Orga wenden. Bitte haltet euch an das KISS-Prinzip\footnote[2]{\url{https://de.wikipedia.org/wiki/KISS-Prinzip}} (Keep it simple, stupid), versucht also eine möglichst einfache Lösung zu erstellen. Es muss auch kein Wunderwerk der Technik sein. Dennoch sind kreative Ideen gerne gesehen.
	
	\subsection*{Die Abgabe} 
	Es gibt zwei Möglichkeiten der Abgabe: Bis spätestens 14:00 Uhr am Donnerstag (03.10.2019) könnt ihr eure \texttt{.py}-Datei in Moodle hochladen oder ihr schickt uns eine Mail an \href{mailto:vorkurs@d120.de}{\nolinkurl{vorkurs@d120.de}}. Dort hängt ihr die Datei bitte als Anhang an. Dabei gilt die Ankunftszeit bei uns.
\end{document}

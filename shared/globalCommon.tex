%Includes
\usepackage[ngerman]{babel} %Deutsche Silbentrennung
\usepackage[utf8]{inputenc} %Deutsche Umlaute
\usepackage{float}
\usepackage{graphicx}
\usepackage{minted}

\DeclareGraphicsExtensions{.pdf,.png,.jpg}

\makeatletter
\author{Vorkursteam der Fachschaft Informatik}
\let\Author\@author

% macros
\renewcommand{\arraystretch}{1.2} % Höhe einer Tabellenspalte minimal erhöhen
\newcommand{\N}{{\mathbb N}}
\newcommand{\code}{\inputminted[]{python}}
\newmintedfile[pythonfile]{python}{
	fontsize=\small,
	style=friendly,
	linenos=true,
	numberblanklines=true,
	tabsize=4,
	obeytabs=false,
	breaklines=true,
	autogobble=true,
	encoding="utf8",
	showspaces=false,
	xleftmargin=20pt,
	frame=single,
	framesep=5pt,
}
\newmintinline{python}{
	style=friendly,
	encoding="utf8"
}

\definecolor{codegray}{HTML}{eaf1ff}
\newminted[bashcode]{awk}{
	escapeinside=||,
	fontsize=\small,
	style=friendly,
	linenos=true,
	numberblanklines=true,
	tabsize=4,
	obeytabs=false,
	breaklines=true,
	autogobble=true,
	encoding="utf8",
	showspaces=false,
	xleftmargin=20pt,
	frame=single,
	framesep=5pt
}

\let\origpythonfile\pythonfile
\renewcommand{\pythonfile}[1]{\pythonfileh{#1}{}}
\newcommand{\pythonfileh}[2]{\origpythonfile[#2]{#1}}

\newcommand*{\ditto}{\texttt{\char`\"}}
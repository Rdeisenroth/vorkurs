\subsection{Mathe 0 für Informatiker*innen}
Wenn euch das Programm aber nur das zurückgeben könnte, was ihr hinschreibt,
dann wäre das ja noch lange kein \textit{Computer}. Darum rechnen wir nun etwas.
Überlegt euch was die folgenden Ausdrücke ergeben, und was die verwendeten
Symbole für Operationen bezeichnen. Beachtet dabei, die Operatorenpräzedenz.

\begin{multicols}{3}
\begin{itemize}
    \item \texttt{16 + 26}
    \item \texttt{13.75 + 28.67}
    \item \texttt{2 * 3 + 6 * 6}
    \item \texttt{'4' + '2'}
    \item \texttt{4 ** 3 - 11 * 2}
    \item \texttt{12 * 133 / 28 - 15}
    \item \texttt{12 * 'a'}
    \item \texttt{1.3 * 5.6}
\end{itemize}
\end{multicols}

\bonus{Mit \texttt{ord(...)} könnt ihr den Codepunkt eines Zeichens
herausfinden, mit \texttt{chr(...)} das Zeichen zu einem Codepunkt.}

%%& -job-name=Aufgaben1_16WS_v1
\documentclass[accentcolor=3c,colorbacktitle,12pt]{tudaexercise}
\usepackage[T1]{fontenc}
%\usepackage[utf8]{inputenc}
%\usepackage[ngerman]{babel}
%Includes
\usepackage[ngerman]{babel} %Deutsche Silbentrennung
\usepackage[utf8]{inputenc} %Deutsche Umlaute
\usepackage{float}
\usepackage{graphicx}
\usepackage{minted}

\DeclareGraphicsExtensions{.pdf,.png,.jpg}

\makeatletter
\author{Vorkursteam der Fachschaft Informatik}
\let\Author\@author

% dark Mode
\ExplSyntaxOn
\RequirePackage{pagecolor,xcolor, graphicx} % Used for dark Mode
\bool_gset_false:N \g_dark_mode_bool % Disable by default
\newcommand{\enableDarkMode}{ %Command to enable Dark Mode (only works before \begin{document})
	\definecolor{anthrazitgrau}{HTML}{293133}
	\pagecolor{anthrazitgrau}
	\color{white}

	\cs_if_exist:NT \setbeamercolor {
		\setbeamercolor*{smallrule}{bg=.}
		\setbeamercolor*{normal~text}{bg=,fg=.}
		\setbeamercolor*{background canvas}{parent=normal~text}
		\setbeamercolor*{section~in~toc}{parent=normal~text}
		\setbeamercolor*{subsection~in~toc}{parent=normal~text,fg=\thepagecolor}
		\setbeamercolor*{footline}{parent=normal~text}
		\setbeamercolor{block~title~alerted}{fg=white,bg=white!20!\thepagecolor}
		\setbeamercolor*{block~body}{bg=white!10!\thepagecolor}
		\setbeamercolor*{block~body~alerted}{bg=\thepagecolor}
	}
	\cs_if_exist:NT \setbeamertemplate {
		\setbeamertemplate{subsection~in~toc~shaded}[default][50]
	}
	\bool_gset_true:N \g_dark_mode_bool
	% Prefer inverted Logo with dark Mode
	% \IfFileExists{tuda_logo_inverted.pdf}{\tl_gset:Nn \g_ptxcd_logofile_tl {tuda_logo_inverted.pdf}}{}
	% \hbox_gset:Nn \g__ptxcd_logo_box {% Update Logo Box
	% 	\makebox[2.2\c_ptxcd_logoheight_dim][l]{\includegraphics[height=\c_ptxcd_logoheight_dim]{\g_ptxcd_logofile_tl}}%
	% }
}

% dark Mode Makros
\prg_new_conditional:Nnn \__ptxcd_if_dark_mode: {T,F,TF} { % Conditional to check if dark Mode is active
	\bool_if:NTF \g_dark_mode_bool
	{\prg_return_true:}
	{\prg_return_false:}
}

\cs_set_eq:NN\IfDarkModeT \__ptxcd_if_dark_mode:T % Easy dark Mode check for use in document
\cs_set_eq:NN\IfDarkModeF \__ptxcd_if_dark_mode:F
\cs_set_eq:NN\IfDarkModeTF \__ptxcd_if_dark_mode:TF

\newcommand{\includeinvertablegraphics}[2][]{% Grafik wird beim Dark Mode automatisch Invertiert (rgb)
	\IfDarkModeTF{\includegraphics[decodearray={1.0~0.0~1.0~0.0~1.0~0.0},#1]{#2}}{\includegraphics[#1]{#2}}
}
\newcommand{\includeinvertablegrayscalegraphics}[2][]{% Grafik wird beim Dark Mode automatisch Invertiert (grayscale)
	\IfDarkModeTF{\includegraphics[decodearray={1.0 0.0},#1]{#2}}{\includegraphics[#1]{#2}}
}

% DARK_MODE environment check (enable if DARK_MODE=1)
\sys_get_shell:nnN { kpsewhich ~ --var-value ~ DARK_MODE } { } \l_dark_mode_env_var_tl
\tl_trim_spaces:N \l_dark_mode_env_var_tl
\tl_if_eq:NnT \l_dark_mode_env_var_tl {1} {\enableDarkMode{}}
\ExplSyntaxOff

% macros
\renewcommand{\arraystretch}{1.2} % Höhe einer Tabellenspalte minimal erhöhen
\newcommand{\N}{{\mathbb N}}
\newcommand{\code}{\inputminted[]{python}}
\newmintedfile[pythonfile]{python}{
	fontsize=\small,
	style=\IfDarkModeTF{fruity}{friendly},
	linenos=true,
	numberblanklines=true,
	tabsize=4,
	obeytabs=false,
	breaklines=true,
	autogobble=true,
	encoding="utf8",
	showspaces=false,
	xleftmargin=20pt,
	frame=single,
	framesep=5pt,
}
\newmintinline{python}{
	style=\IfDarkModeTF{fruity}{friendly},
	encoding="utf8"
}

\definecolor{codegray}{HTML}{eaf1ff}
\newminted[bashcode]{awk}{
	escapeinside=||,
	fontsize=\small,
	style=\IfDarkModeTF{fruity}{friendly},
	linenos=true,
	numberblanklines=true,
	tabsize=4,
	obeytabs=false,
	breaklines=true,
	autogobble=true,
	encoding="utf8",
	showspaces=false,
	xleftmargin=20pt,
	frame=single,
	framesep=5pt
}

\let\origpythonfile\pythonfile
\renewcommand{\pythonfile}[1]{\pythonfileh{#1}{}}
\newcommand{\pythonfileh}[2]{\origpythonfile[#2]{#1}}

\newcommand*{\ditto}{\texttt{\char`\"}}
\usepackage{hyperref}
\usepackage{ifthen}
\usepackage{listings}
%\usepackage{graphicx}
\usepackage{multicol}
\usepackage{multirow}
\usepackage{amssymb}


\definecolor{darkblue}{rgb}{0,0,.5}
\hypersetup{colorlinks=true, breaklinks=true, linkcolor=darkblue, menucolor=darkblue, urlcolor=darkblue}

% configuration
\makeatletter
\@ifundefined{c@ex}{
  \newcounter{ex}\setcounter{ex}{1}
}{}
\makeatother
\newboolean{sln}\setboolean{sln}{false}
\newboolean{SoSe}\setboolean{SoSe}{false}
\newcounter{nextyear}\setcounter{nextyear}{\year+1}
%

\newcommand{\ext}{py}
\newcommand{\sln}[1]{\ifthenelse{\boolean{sln}}{\subsubsection*{Antwort}{\itshape #1}}{}}
\newcommand{\slnformat}[1]{\ifthenelse{\boolean{sln}}{#1}{}}
\newcommand{\vorkurstaskformat}[1]{\ifthenelse{\boolean{sln}}{}{#1}}
\newcommand{\vorkurstask}[1]{\input{task/#1}\IfFileExists{./sln/#1.tex}{\sln{\input{sln/#1}}}{\IfFileExists{./sln/#1.\ext}{\sln{\pythonfile{sln/#1.\ext}}}{\ClassError{Vorkurs-TeX}{No solution specified for task #1}{Add solution file #1.tex or #1.\ext}}}}
\newcommand{\mccmd}{Kreuze zu jeder Antwort an, ob sie zutrifft (\textbf{w}) oder nicht (\textbf{f}).}
\newcommand{\mchead}{\item[\textbf{w} \textbf{f} ]}
\newcommand{\mcitem}[1]{\item[$\square\ \square$] #1}
\newcommand{\mcitemt}[1]{\item[$\ifthenelse{\boolean{sln}}{\blacksquare}{\square}\ \square$] #1}
\newcommand{\mcitemf}[1]{\item[$\square\ \ifthenelse{\boolean{sln}}{\blacksquare}{\square}$] #1}
\newcommand{\ptitle}{\ifthenelse{\boolean{SoSe}}{Sommersemester \the\year}{Wintersemester \the\year/\thenextyear}}

\newcommand{\lstinlinenoit}[1]{\upshape{\lstinline|#1|}\itshape}
\lstset{language=Python, basicstyle=\ttfamily\small, keywordstyle=\color{blue!80!black}, identifierstyle=, commentstyle=\color{green!50!black}, stringstyle=\ttfamily,
 tabsize=4, breaklines=true, numbers=left, numberstyle=\small, frame=single, backgroundcolor=\color{blue!3}}
\author{Fachschaft Informatik}


\newcommand{\stage}[1]{(\ifcase#1\or{Einstieg}\or{Vertiefung}\or{Herausforderung}\else\fi)}
\newcommand{\bonus}[1]{\textit{BonusFact: }#1}

\begin{document}
\title{Aufgaben Programmiervorkurs\\Übungsblatt \theex}
\subtitle{von der Fachschaft Informatik\hfill\ptitle}
\maketitle

\ifcase\value{ex}
\or%ex1
    \section{Einleitung \stage1}
        \vorkurstask{1-intro}
        \vorkurstask{1-mc-interpreter}\vorkurstaskformat{\clearpage}
    \section{Ausdrücke \stage1}
        \vorkurstask{1-interactive}
        \vorkurstask{1-operations}\vorkurstaskformat{\clearpage}\slnformat{\clearpage}
    \section{Konvertierung \stage2}
        \vorkurstask{1-conversion}
        \vorkurstask{1-conv-challenge}\slnformat{\clearpage}
	\section{Fehler}
        \vorkurstask{1-error}
		\vorkurstask{1-warmup-error}
    \section{Challenge}
        \vorkurstask{1-modulo}
\or%ex2
        \vorkurstask{2-pyfile}
	\section{Zum Aufwärmen \stage1}\mccmd
		\vorkurstask{2-mc-if}
	\section{Variablen \stage1}
		\vorkurstask{2-var-naming}\slnformat{\clearpage}
		\vorkurstask{2-var-id}\vorkurstaskformat{\clearpage}
		\vorkurstask{2-var-assign}\slnformat{\clearpage}
	\section{Logische Operationen \stage1}
		\vorkurstask{2-stm-ops}\vorkurstaskformat{\clearpage}
	\section{Eingabe/Ausgabe \stage2}
		\vorkurstask{2-var-io}
		\vorkurstask{2-stm-calc}\vorkurstaskformat{\clearpage}\slnformat{\clearpage}
	\section{Zum Weiterdenken \stage3}
		\vorkurstask{2-cooking}\vorkurstaskformat{\clearpage}\slnformat{\clearpage}
		\vorkurstask{2-var-processor}
\or%ex3
    \section{Schleifen}
		\vorkurstask{3-loop-gauss}
        \vorkurstask{3-loop-fac}
        \vorkurstask{3-loop-fizz}\vorkurstaskformat{\clearpage}\slnformat{\clearpage}
        \vorkurstask{3-loop-prime}\slnformat{\clearpage}
	\section{Listige Listen}
		\vorkurstask{3-list-basic}
        \vorkurstask{3-list-len}
	\section{Fortgeschrittenes Lesen}
        \vorkurstask{3-slicing}
        \vorkurstask{3-slicing-examples}\slnformat{\clearpage}
	\section{Listenverarbeitung}
        \vorkurstask{3-iterable}
        \vorkurstask{3-text-slicing}\slnformat{\clearpage}
        \vorkurstask{3-accumulator}
	\section{Challenge}
		\vorkurstask{3-challenge}
\or%ex4
	\section{Funktionen}
		\vorkurstask{4-fnc-calcr}\slnformat{\clearpage}
		\vorkurstask{4-fnc-calcr2}\slnformat{\clearpage}
	\section{Schriftliches Dividieren}
		\vorkurstask{4-fnc-divide-theory}\slnformat{\clearpage}
		\vorkurstask{4-fnc-divide-basic}\slnformat{\clearpage}
		\vorkurstask{4-fnc-divide-advanced}\slnformat{\clearpage}
	\section{Listen und Funktionen}
		\vorkurstask{4-list-fnc}\slnformat{\clearpage}
	\section{Rekursion}
		\vorkurstask{4-rec-fac}
		\vorkurstask{4-rec-pascal}\slnformat{\clearpage}
		\vorkurstask{4-rec-fib}\slnformat{\clearpage}
		\vorkurstask{4-rec-oddeven}\slnformat{\clearpage}
		\vorkurstask{4-rec-calcu}\vorkurstaskformat{\clearpage}
		\vorkurstask{4-rec-calcu2}
	\section{Zusatz}
		Wenn du die Aufgaben erledigt hast, kannst du dich gerne noch an den Schleifen-Varianten der gegebenen Algorithmen versuchen oder für Aufgaben der vergangenen Tage eine rekursive Lösung suchen. Die Tutoren stehen dir dabei gerne bei, allerdings werden wir dafür keine Lösungen bereit stellen. Du kannst die jeweiligen Versionen der Algorithmen in den Lösungen als Kontrolle für deine Versuche nutzen.
\or%ex5 - nicht veröffentlicht
	\section{Pool}
	\section{Sichtbarkeit}
	\section{Ausdrücke evaluieren}
\else\fi
\end{document}
\grid
\grid

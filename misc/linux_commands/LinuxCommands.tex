% !TeX document-id = {505cbeed-ac21-4fdd-a7f4-eefb28b91c1e}
% !TeX TXS-program:compile = txs:///pdflatex/[--shell-escape]
\documentclass[accentcolor=tud3c,colorbacktitle,12pt]{tudexercise}
%Includes
\usepackage[ngerman]{babel} %Deutsche Silbentrennung
\usepackage[utf8]{inputenc} %Deutsche Umlaute
\usepackage{float}
\usepackage{graphicx}
\usepackage{minted}

\DeclareGraphicsExtensions{.pdf,.png,.jpg}

\makeatletter
\author{Vorkursteam der Fachschaft Informatik}
\let\Author\@author

% dark Mode
\ExplSyntaxOn
\RequirePackage{pagecolor,xcolor, graphicx} % Used for dark Mode
\bool_gset_false:N \g_dark_mode_bool % Disable by default
\newcommand{\enableDarkMode}{ %Command to enable Dark Mode (only works before \begin{document})
	\definecolor{anthrazitgrau}{HTML}{293133}
	\pagecolor{anthrazitgrau}
	\color{white}

	\cs_if_exist:NT \setbeamercolor {
		\setbeamercolor*{smallrule}{bg=.}
		\setbeamercolor*{normal~text}{bg=,fg=.}
		\setbeamercolor*{background canvas}{parent=normal~text}
		\setbeamercolor*{section~in~toc}{parent=normal~text}
		\setbeamercolor*{subsection~in~toc}{parent=normal~text,fg=\thepagecolor}
		\setbeamercolor*{footline}{parent=normal~text}
		\setbeamercolor{block~title~alerted}{fg=white,bg=white!20!\thepagecolor}
		\setbeamercolor*{block~body}{bg=white!10!\thepagecolor}
		\setbeamercolor*{block~body~alerted}{bg=\thepagecolor}
	}
	\cs_if_exist:NT \setbeamertemplate {
		\setbeamertemplate{subsection~in~toc~shaded}[default][50]
	}
	\bool_gset_true:N \g_dark_mode_bool
	% Prefer inverted Logo with dark Mode
	% \IfFileExists{tuda_logo_inverted.pdf}{\tl_gset:Nn \g_ptxcd_logofile_tl {tuda_logo_inverted.pdf}}{}
	% \hbox_gset:Nn \g__ptxcd_logo_box {% Update Logo Box
	% 	\makebox[2.2\c_ptxcd_logoheight_dim][l]{\includegraphics[height=\c_ptxcd_logoheight_dim]{\g_ptxcd_logofile_tl}}%
	% }
}

% dark Mode Makros
\prg_new_conditional:Nnn \__ptxcd_if_dark_mode: {T,F,TF} { % Conditional to check if dark Mode is active
	\bool_if:NTF \g_dark_mode_bool
	{\prg_return_true:}
	{\prg_return_false:}
}

\cs_set_eq:NN\IfDarkModeT \__ptxcd_if_dark_mode:T % Easy dark Mode check for use in document
\cs_set_eq:NN\IfDarkModeF \__ptxcd_if_dark_mode:F
\cs_set_eq:NN\IfDarkModeTF \__ptxcd_if_dark_mode:TF

\newcommand{\includeinvertablegraphics}[2][]{% Grafik wird beim Dark Mode automatisch Invertiert (rgb)
	\IfDarkModeTF{\includegraphics[decodearray={1.0~0.0~1.0~0.0~1.0~0.0},#1]{#2}}{\includegraphics[#1]{#2}}
}
\newcommand{\includeinvertablegrayscalegraphics}[2][]{% Grafik wird beim Dark Mode automatisch Invertiert (grayscale)
	\IfDarkModeTF{\includegraphics[decodearray={1.0 0.0},#1]{#2}}{\includegraphics[#1]{#2}}
}

% DARK_MODE environment check (enable if DARK_MODE=1)
\sys_get_shell:nnN { kpsewhich ~ --var-value ~ DARK_MODE } { } \l_dark_mode_env_var_tl
\tl_trim_spaces:N \l_dark_mode_env_var_tl
\tl_if_eq:NnT \l_dark_mode_env_var_tl {1} {\enableDarkMode{}}
\ExplSyntaxOff

% macros
\renewcommand{\arraystretch}{1.2} % Höhe einer Tabellenspalte minimal erhöhen
\newcommand{\N}{{\mathbb N}}
\newcommand{\code}{\inputminted[]{python}}
\newmintedfile[pythonfile]{python}{
	fontsize=\small,
	style=\IfDarkModeTF{fruity}{friendly},
	linenos=true,
	numberblanklines=true,
	tabsize=4,
	obeytabs=false,
	breaklines=true,
	autogobble=true,
	encoding="utf8",
	showspaces=false,
	xleftmargin=20pt,
	frame=single,
	framesep=5pt,
}
\newmintinline{python}{
	style=\IfDarkModeTF{fruity}{friendly},
	encoding="utf8"
}

\definecolor{codegray}{HTML}{eaf1ff}
\newminted[bashcode]{awk}{
	escapeinside=||,
	fontsize=\small,
	style=\IfDarkModeTF{fruity}{friendly},
	linenos=true,
	numberblanklines=true,
	tabsize=4,
	obeytabs=false,
	breaklines=true,
	autogobble=true,
	encoding="utf8",
	showspaces=false,
	xleftmargin=20pt,
	frame=single,
	framesep=5pt
}

\let\origpythonfile\pythonfile
\renewcommand{\pythonfile}[1]{\pythonfileh{#1}{}}
\newcommand{\pythonfileh}[2]{\origpythonfile[#2]{#1}}

\newcommand*{\ditto}{\texttt{\char`\"}}

\title{Programmiervorkurs Linux-Befehle}
\subtitle{\Author}
\subsubtitle{vorkurs@d120.de}

\begin{document}
\maketitle

\section{Dateisystem}

\subsection{cd}
In ein anderes Verzeichnis wechseln.
\begin{bashcode}
cd |\textit{Pfad zum Verzeichnis}|
\end{bashcode}


\subsection{ls}
Listet alle Dateien im aktuellen Verzeichnis auf.
\begin{bashcode}
ls
\end{bashcode}
Listet alle Dateien im aktuellen Verzeichnis in Listenform auf.
\begin{bashcode}
ls -l
\end{bashcode}

\subsection{mv}
Verschiebt oder benennt eine Datei oder ein Verzeichnis um.
\begin{bashcode}
mv "Name" "neuer Name" 
mv "Pfad/Name" "neuer Pfad/neuer Name" 
\end{bashcode}

\subsection{mkdir}
Erstellt ein neues Verzeichnis.
\begin{bashcode}
mkdir |\textit{Ordnername}|
\end{bashcode}

\subsection{cp}
Kopiert eine Datei.\\
Kopiert einen Ordner mit Option \textit{-R}.
\begin{bashcode}
cp |\textit{Datei}| |\textit{Kopie}|
cp -R |\textit{Ordner}| |\textit{KopierterOrdner}|
\end{bashcode}

\subsection{rm}
Löscht eine Datei oder mit Option -R einen Ordner.
\begin{bashcode}
rm |\textit{Datei}|
rm -R |\textit{Ordner}|
\end{bashcode}

\section{Programmieren}
\subsection{nano und gedit}
Nano ist ein Konsolen-Texteditor. Gibt man als Parameter eine existierende Datei an, wird diese geöffnet; existiert die Datei nicht, wird sie beim Speichern erstellt. Gedit ist quasi das gleiche mit graphischer Oberfläche.
\begin{bashcode}
nano |\textit{Datei}|
gedit |\textit{Datei}|
\end{bashcode}
Innerhalb des Editors kann man mit den Pfeiltasten navigieren und wie gewohnt schreiben.\\
Mit Strg+O kann man die Datei speichern, man wird nach dem Pfad zum Speichern gefragt und kann diesen mit Enter bestätigen.\\
Mit Strg+X verlässt man den Editor.

\subsection{Python}
\subsubsection*{Linux/macOS}
Mit \textit{python3} kann man py-Skripte ausführen, wobei Python 3.x.x. verwendet wird. \textit{python} wiederum verwendet 2.x.x und ist veraltet, weshalb es nicht mehr verwendet werden sollte.
\begin{bashcode}
python3 |\textit{Datei.py}|
\end{bashcode}

\subsubsection*{Windows}
Mit \textit{python} kann man py-Skripte ausführen.
\begin{bashcode}
python |\textit{Datei.py}|
\end{bashcode}

\end{document}
